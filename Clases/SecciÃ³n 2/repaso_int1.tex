\documentclass{beamer}
\usepackage[spanish]{babel}
\usepackage{tabularx,colortbl}
\usepackage{amsmath,amsthm,amsfonts,amssymb,stmaryrd}
\usepackage{tikz}
\usepackage{pgfplots}
\usetikzlibrary{positioning}
%Information to be included in the title page:
\title{IIC1253 Matemáticas Discretas}
\author{Sasha Kozachinskiy}
\institute{DCC UC}
\date{29.09.2025}
\newtheorem{teorema}{Teorema}
\newtheorem{proposicion}{Proposición}
\newtheorem{corolario}{Corolario}
\newtheorem{definicion}{Definición}
\newtheorem{notacion}{Notación}
\newtheorem{lema}{Lema}
\newtheorem{problema}{Problema}
\newtheorem{axioma}{Axioma}
\newcommand{\suma}{\mathsf{suma}}
\newcommand{\producto}{\mathsf{producto}}
\begin{document}

\begin{frame}
\end{frame}

\begin{frame}
\end{frame}

\begin{frame}
\end{frame}

\begin{frame}
\end{frame}

\begin{frame}
\end{frame}

\begin{frame}
\end{frame}

\begin{frame}
\end{frame}

\begin{frame}
\end{frame}

\begin{frame}
\end{frame}

\begin{frame}
\end{frame}

\begin{frame}
\end{frame}

\begin{frame}
\end{frame}

\begin{frame}
\end{frame}

\begin{frame}
\end{frame}

\begin{frame}
\end{frame}

\begin{frame}
\end{frame}

\begin{frame}
\end{frame}

\begin{frame}
\end{frame}

\begin{frame}
\end{frame}

\begin{frame}
\end{frame}

\begin{frame}
\end{frame}

\begin{frame}
\end{frame}

\begin{frame}
\end{frame}

\begin{frame}
\end{frame}
\begin{frame}
\end{frame}

\begin{frame}
\end{frame}

\begin{frame}
\end{frame}

\begin{frame}
\end{frame}

\begin{frame}
\end{frame}

\begin{frame}
\end{frame}

\begin{frame}
\end{frame}

\begin{frame}
\end{frame}


\end{document}

\frame{\titlepage}
\begin{frame}{Hoy...}
\huge
inducción matemática: números naturales, inducción, orden.

\end{frame}

%\begin{frame}{¿Conjuntos infinitos?}
%\begin{itemize}
%\pause
%\item Nada garantiza existencia de conjuntos infinitos.
%
%\pause
%\item Plan: definir números naturales...
%
%\pause
%\item ...y \emph{conjunto} de los números naturales...
%
%\pause
%\item ...tal que el principio  de inducción es cierto.
%\end{itemize}
%\end{frame}

\begin{frame}{Operación sucesor}
\begin{definicion}Sea $a$ un conjunto. Su \textbf{sucesor} es el conjunto $S(a) = a\cup \{a\}$.
\end{definicion}

\pause

Ejercicio: escribir una fórmula que expresa ``$b = S(a)$'':

\pause

\vspace{1cm}

Primeros números naturales:

\[0 = \varnothing\]

\[1 = S(\varnothing) = \]

\[2 = S(1) = \]


\[3 = S(2) = \]
\end{frame}

\begin{frame}{Conjuntos inductivos y infinitud}
\begin{definicion} Un conjunto $I$ se llama \textbf{inductivo} si 
\begin{enumerate}
\item[a)]  $\varnothing \in I$;

\item[b)] para cada $x\in I$, tenemos $S(x) \in I$.
\end{enumerate}
\end{definicion}

\pause

\begin{axioma}[de infinitud]
Existe un conjunto inductivo.
\end{axioma}
\end{frame}


\begin{frame}{Números naturales}
\begin{definicion} Un conjunto $n$ se llama un \textbf{número natural} si $n\in I$ para todos los conjuntos inductivos $I$.
\end{definicion}
\pause

Ejercicio: escribir una fórmula ``$n$ es un número natural''.

\pause

\begin{proposicion} $0 = \varnothing$ es un número natural, y si $n$ es un número natural, entonces $S(n)$ tambíen es un número natural.
\end{proposicion}

\vspace{4cm}
\end{frame}


\begin{frame}{Conjunto de números naturales y principio de inducción}
\begin{teorema}
Existe un conjunto $\mathbb{N}$ de todos los números naturales.
\end{teorema}
\pause
\begin{proof} $\mathbb{N} = \{n\in I\mid \text{$n$ es un número natural}\}$ para cualquier conjunto inductivo $I$.
\end{proof}

\pause

\begin{teorema}[Principio de inducción]
Sea $A$ un subconjunto de $\mathbb{N}$ tal que 
\begin{enumerate}
\item[a)]  $0  \in A$;

\item[b)] para cada $n\in A$, tenemos $S(n) = n + 1 \in A$.
\end{enumerate}
Entonces, $A = \mathbb{N}$.
\end{teorema}
\pause 
\begin{proof} $A$ es inductivo, por lo tanto $\mathbb{N}\subseteq A$. Ya que $A\subseteq \mathbb{N}$, obtenemos $A = \mathbb{N}$.
\end{proof}

\end{frame}


\begin{frame}{Orden}

\begin{definicion} Sean $n,m$ dos números naturales. Entonces, $n<m$ si $n\in m$.
\end{definicion}

\begin{teorema}[Propiedades de orden]
\begin{itemize}
\item[a)] $\lnot (n < n)$ para todos los números naturales $n$;
\item[b)] $n < S(n)$ para todos los números naturales $n$;
\item[c)] $0 < n$ o $0 = n$ para todos los números naturales $n$;
\item[d)] $((n < m) \land (m < k)) \to (n < k)$ para todos los números naturales $n,m,k$;
\item[e)] $(n < m) \lor (m < n) \lor (n = m)$ para todos los números naturales $n,m$;
\item[f)] no existen dos números naturales $n,m$ tal que $n < m < S(n)$.
\end{itemize}
\end{teorema}
\end{frame}

\begin{frame}
\begin{proposicion}  $0 < n$ o $0 = n$ para todos los números naturales $n$.
\end{proposicion}




\vspace{5cm}
\end{frame}


\begin{frame}
\begin{proposicion}  $((n < m) \land (m < k)) \to (n < k)$ para todos los números naturales $n,m,k$;
\end{proposicion}

\pause 

\begin{definicion} Un conjunto $a$ se llama \textbf{transitivo} si $c\in b \in a \implies c\in a$ para todos los conjuntos $b,c$.
\end{definicion}

\pause

\begin{proposicion}  Todos los números naturales son transitivos.
\end{proposicion}

\vspace{2cm}
\end{frame}

\begin{frame}
\end{frame}
\begin{frame}%{Conclusión}

%Se puede definir la suma y el producto cómo subconjuntos de $\mathbb{N}^3$, números racionales cómo pares, números reales cómo \emph{funciones} $f\colon\mathbb{N} \to\mathbb{Q}$, etc...

%\pause
\begin{center}
\huge
¡Gracias!
\end{center}
\end{frame}
\end{document}

